%% Configurações de aparência do PDF final

% ---
% Configurações do pacote backref
% Texto padrão antes do número das páginas
\renewcommand{\backref}{}
% Define os textos da citação
\renewcommand*{\backrefalt}[4]{
	\ifcase #1
		Nenhuma citação no texto.
	\or
		Citado na página #2.
	\else
		Citado #1 vezes nas páginas #2.
	\fi}
% ---

%% Cor dos Links do PDF
%% Esta é a cor azul:
%\definecolor{cor-link}{RGB}{8,40,75}
%% Esta é a cor preta ("esconde" os links)
\definecolor{cor-link}{RGB}{0,0,0}

%% Cor para os quadros, se usar. Dê preferência a cores escuras.
%% Boa referência para cores: https://material.io/guidelines/style/color.html#color-color-palette
\definecolor{cor-quadro}{RGB}{5,28,63}

%% informações do PDF
\makeatletter
\hypersetup{
  pdftitle={\@title}, 
  pdfauthor={\@author},
  pdfsubject={\@title},
  pdfcreator={Overleaf, LaTeX, abnTeX2},
  pdfkeywords={financiamento coletivo}{empreendimento social}{crowdfunding}{impacto social}{social impact}{social entrepreneurship},
  colorlinks=true,							% Visual dos Links: false = caixas; true = colorido
  linkcolor=cor-link,							% Cor dos Links Internos (preto)
  citecolor=cor-link,							% Cor de Links para Bibliografia (preto)
  filecolor=cor-link,							% Cor para Links a Arquivos (preto)
  urlcolor=cor-link,							% Cor para Links a URLs (preto)
  bookmarksdepth=4
}
\makeatother



% O tamanho do parágrafo é dado por:
\setlength{\parindent}{1.5cm}



% Controle do espaçamento entre um parágrafo e outro:
\setlength{\parskip}{0.2cm}  % tente também \onelineskip



%% compila o sumário e glossário
%\makeglossaries
\makeindex