% ----------------------------------------------------------
% Fundamentação Teórica
% ----------------------------------------------------------
\chapter{Fundamentação Teórica}

\section{Crowdsourcing}
% Definição
Antes de se falar de Crowdfunding é importante conhecer o movimento o qual originou. Crowdsourcing foi definido por \citeauthor{wired-crowdsource} em \citeyear{wired-crowdsource} como o ato de uma companhia ou instituição pegar uma função desempenhada por seus funcionários e terceiriza-la para uma rede anônima geralmente bastante grande de pessoas na forma de uma chamada aberta.

Isso pode tomar a forma de um sistema cooperativo, onde o trabalho é feito de forma colaborativa, mas também frequentemente o trabalho é executado por indivíduos. O prerrequisito crucial é o uso de um formato de chamada aberta e a grande rede de potenciais trabalhadores para atendê-lo.

% Exemplos
Essa modalidade de terceirização não é recente. Por exemplo em 1714 o Governo Britânico conduziu um concurso chamado Prêmio Longitude\cite{wiki-longitude_rewards}, que daria como recompensa 10 a 20 mil libras, dependendo da qualidade da solução, para quem resolvesse um dos maiores problemas da época: como determinar a longitude de uma embarcação em alto-mar.

Um segundo exemplo notável é o de \citeauthor{hearn2002tracks} que conta como Matthew Fontaine Maury, em 1848, distribuiu de forma gratuita 5000 cópias de seu catálogo de correntes marítimas e eólicas pedindo em troca que os marinheiros fizessem e entregassem seus diários de bordo ao retornar de suas jornadas.

Com avanços nas técnicas de desenvolvimento de software, computação distribuída, processamento de dados e a popularização da Internet a partir da década de 1990 projetos como o \emph{Great Internet Mersenne Prime Search}\footnote{Projeto para buscar Primos de Mersenne (números resultantes de \( 2^p - 1 \)) - http://www.mersenne.org} (GIMPS), lançado em 1996, e o SETI@Home\footnote{Análise de ondas de rádio cósmicas em busca de vida extra terrestre - https://setiathome.berkeley.edu}, lançado em 1999, começaram a utilizar do poder de processamento latente nos computadores pessoais de inúmeros voluntários ao redor do planeta para executarem tarefas massivas que, sem isso, necessitariam de supercomputadores e bastante investimento.

Naturalmente esse aspecto da coletivização da ajuda evoluiu para além da realização de tarefas, grandes ou pequenas, e surgiram novas modalidades de Crowdsourcing, a mais prominente sendo o Crowdfunding.



\section{Crowdfunding}
\citeauthor{belleflamme2010} definem crowdfunding como a utilização do crowdsourcing para captação de dinheiro para investimento geralmente utilizando redes sociais, em particular as via Internet, com ou sem a expectativa de retorno por parte do investidor.

Para \citeauthor{golan2015crowdfunding} o papel da comunidade é peça chave no funcionamento dessa modalidade de financiamento e a Internet indispensável para a formação dessas comunidades. Pessoas com interesses semelhantes facilmente se agregam e se organizam através de redes sociais, fóruns e outras ferramentas em comunidades. Através dessas comunidades que os potenciais apoiadores serão alcançados.

Campanhas de crowdfunding comumente utilizam-se de duas formas para incentivar os apoiadores \cite{belleflamme2014crowdfunding}. A primeira é através de uma espécie de venda ou pré-venda de algum produto, geralmente mostrado como uma recompensa pela ajuda ao projeto. A segunda é através da venda de participação na empresa a qual promove a campanha (ou que será criada em decorrência da campanha).

\citeauthor{lehner2013crowdfunding} julga que a venda de participação, apesar de mais complexa e arriscada, pode ser beneficial no contexto de empreendimentos sociais pois o nível de engajamento da comunidade pode servir como validação da ideia por trás do empreendimento. Além disso, um dos fatores motivacionais para apoiadores de projetos socais é a participação do mesmo nas ações daquela empresa ou organização.

Entretanto dada a complexidade da criação de uma empresa sem fins lucrativos e regulamentação necessária para abertura de patrimônio esta se torna uma opção inviável para as organizações que este trabalho propõe a atender.

Tendo isso em consideração fatores como validação do trabalho da ONG podem ser medidos pelo engajamento voluntário através da análise do volume e valor arrecadado estritamente na forma de doações que não acarretarão em um retorno direto, ou seja, nenhum produto ou serviço é oferecido ao apoiador do projeto. Dessa forma as motivações para ajuda ao projeto é a identificação do indivíduo com o trabalho proposto pela organização.

--- Falar mais coisas. Que direção tomar o texto nessa seção? ---



\section{Empreendedorismo Social}
Falar de empreendedorismo social de forma técnica.

Entretanto... é realmente necessário? O sistema em si não é um empreendimento social, mas quem irá usar é. Não creio que eu faça algo no domínio do empreendedorismo social.



\section*{Resumo}
Neste capítulo foi apresentada a base teórica utilizada para idealização da solução em software, que pode-se resumir em uma ferramenta para tornar a comunicação com uma comunidade e arrecadação de fundos da mesma.

Os próximos capítulos estão organizados da seguinte forma:

\begin{itemize}
  \item \textbf{Metodologia:} Neste capítulo são detalhados os artefatos produzidos nas fases de levantamento de requisitos, modelagem e implementação do processo de software utilizado no desenvolvimento da solução.
  \item \textbf{Ajuda.Ai:} Capítulo dedicado a apresentação da solução implementada, detalhando funcionalidades, utilização e impressões de profissionais do empreendedorismo social quanto a ferramenta.
  \item \textbf{Conclusão:} Nesta seção é feita a conclusão do trabalho dado seus objetivos propostos.
  \item \textbf{Trabalhos Futuros:} Nesta seção são listados os trabalhos futuros para melhorar e/ou expandir a utilização da solução.
\end{itemize}