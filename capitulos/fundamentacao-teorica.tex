% ----------------------------------------------------------
% Fundamentação Teórica
% ----------------------------------------------------------
\chapter{Fundamentação Teórica} \label{cha:fundamentacao}

Para melhor compreender as decisões de projeto tomadas durante o desenvolvimento da solução proposta por este trabalho é necessário entender alguns conceitos teóricos relacionados a \emph{Crowdsourcing} e \emph{Crowdfunding}.

\section{\emph{Crowdsourcing}} \label{sec:fundamentacao:crowdsourcing}
% Definição
Antes de se falar de \emph{Crowdfunding} é importante conhecer o movimento o qual o originou. \emph{Crowdsourcing} foi definido por \citeauthor{wired-crowdsource} em \citeyear{wired-crowdsource} como o ato de uma companhia ou instituição pegar uma função desempenhada por seus funcionários e terceirizá-la para uma rede anônima, geralmente bastante grande, de pessoas na forma de uma chamada aberta. Isso pode tomar a forma de um sistema cooperativo, onde o trabalho é feito de forma colaborativa, mas também frequentemente o trabalho é executado por indivíduos. O pré-requisito crucial é o uso de um formato de chamada aberta e a grande rede de potenciais trabalhadores para atendê-lo.

% Exemplos
Essa modalidade de terceirização não é recente. Por exemplo, em 1714, o Governo Britânico conduziu um concurso chamado Prêmio Longitude\cite{wiki-longitude_rewards}, que daria como recompensa 10 a 20 mil libras, dependendo da qualidade da solução, para quem resolvesse um dos maiores problemas da época: como determinar a longitude de uma embarcação em alto-mar. Um segundo exemplo notável é o de \citeauthor{hearn2002tracks} que conta como Matthew Fontaine Maury, em 1848, distribuiu de forma gratuita 5000 cópias de seu catálogo de correntes marítimas e eólicas pedindo em troca que os marinheiros fizessem e entregassem seus diários de bordo ao retornar de suas jornadas.

Além desses projetos e, mais recentemente, com avanços nas técnicas de desenvolvimento de software, computação distribuída, processamento de dados e a popularização da Internet a partir da década de 1990 projetos como o GIMPS\footnote{\emph{Great Internet Mersenne Prime Search}, Projeto para buscar Primos de Mersenne - http://www.mersenne.org}, lançado em 1996, e o SETI@Home\footnote{Análise de ondas de rádio cósmicas em busca de vida extra terrestre - https://setiathome.berkeley.edu}, lançado em 1999. Esses projetos começaram a utilizar poder de processamento latente nos computadores pessoais de inúmeros voluntários ao redor do planeta para executarem tarefas massivas que, sem isso, necessitariam de supercomputadores e bastante investimento. Através de projetos como esses o conceito de \emph{Crowdsourcing} na era digital foi apresentado e validado, dado que ambos projetos apresentados ainda estão em pleno funcionamento.

Com o tempo, esse aspecto da coletivização da ajuda evoluiu para além da realização de tarefas, grandes ou pequenas, e surgiram novas modalidades de \emph{Crowdsourcing}. Dentre elas, a mais prominente e que mais cresce no mundo é o \emph{Crowdfunding}, onde a tarefa em questão é a arrecadação de financiamento para um determinado fim, como descrito a seguir.



\section{Financiamento Coletivo: \emph{Crowdfunding}} \label{sec:fundamentacao:financiamento}
\citeauthor{belleflamme2010} definem \emph{crowdfunding} como a utilização do \emph{crowdsourcing} para arrecadar dinheiro para um empreendimento. Além disso, eles definem ainda que essa interação é feita via Internet, majoritariamente através de redes sociais, e podendo ou não haver um retorno ao investidor.

Para \citeauthor{golan2015crowdfunding}, o papel da comunidade é peça chave no funcionamento dessa modalidade de financiamento e a Internet indispensável para a formação dessas comunidades. Pessoas com interesses semelhantes facilmente se agregam e se organizam através de redes sociais, fóruns e outras ferramentas em comunidades. É através dessas comunidades que os potenciais apoiadores serão alcançados.

Campanhas de \emph{crowdfunding} comumente se utilizam de duas formas para incentivar os apoiadores\cite{belleflamme2014crowdfunding}. A primeira é através de uma espécie de venda ou pré-venda de algum produto, geralmente mostrado como uma recompensa pela ajuda ao projeto. A segunda é através da venda de participação na empresa a qual promove a campanha ou que será criada em decorrência da campanha.

\citeauthor{lehner2013crowdfunding} julga que a venda de participação, apesar de mais complexa e arriscada, pode ser beneficial no contexto de empreendimentos sociais, pois o nível de engajamento da comunidade pode servir como validação da ideia por trás do empreendimento. Além disso, um dos fatores motivacionais para apoiadores de projetos socais é a participação dos mesmos nas ações daquela empresa ou organização.

Entretanto, dada a complexidade da criação de uma empresa sem fins lucrativos e regulamentação necessária para abertura de patrimônio, essa se torna uma opção inviável para as organizações não-governamentais que este trabalho propõe a atender. Tendo isso em consideração, fatores como validação do trabalho da ONG podem ser medidos pelo engajamento voluntário através da análise do volume e valor arrecadado estritamente na forma de doações que não acarretarão em um retorno direto, ou seja, nenhum produto ou serviço é oferecido ao apoiador do projeto. Dessa forma a motivação para ajuda ao projeto é a identificação do indivíduo com o trabalho proposto pela organização.

Utilizando o poder do \emph{Crowdsourcing} e \emph{Crowdfunding}, é proposta a ferramenta Ajuda.Ai para a arrecadação de investimento através dessa modalidade, com custos reduzidos e boa integração com plataformas de redes sociais provendo também um canal de comunicação com os doadores e comunidade interessada a fim de atender as necessidades e peculiaridades relacionadas a \emph{crowdfunding} para empreendimentos sociais.



\section*{Resumo}
Neste capítulo foi apresentada a base teórica utilizada para idealização da solução em software, que pode-se resumir em uma ferramenta para tornar a comunicação com uma comunidade e arrecadação de fundos da mesma.

No próximo capítulo será apresentada a ferramenta de código-fonte aberto que o trabalho propõe para atender a essas necessidades, de código fonte aberto e chamada Ajuda.Ai.