% ----------------------------------------------------------
% Metodologia
% ----------------------------------------------------------
\chapter{Metodologia}

O desenvolvimento da aplicação utilizou-se de uma metodologia de desenvolvimento de software composta das seguintes fases: Levantamento de Requisitos, Modelagem, Implementação e Testes \cite{sommerville2003engenharia}.



\section*{Resumo}
Neste capítulo foi apresentada uma contextualização sobre o problema tratado neste trabalho e a justificativa de tal assunto, que pode-se resumir como sendo necessidade de uma alternativa moderna e simplificada para captação de recursos para ONGs. Ao final, foram detalhados os objetivos gerais e específicos do trabalho.

Os próximos capítulos estão organizados da seguinte forma:

\begin{itemize}
  \item \textbf{Ajuda.Ai:} Capítulo dedicado a apresentação da solução implementada, detalhando funcionalidades, utilização e impressões de profissionais do empreendedorismo social quanto a ferramenta.
  \item \textbf{Conclusão:} Nesta seção é feita a conclusão do trabalho dado seus objetivos propostos.
  \item \textbf{Trabalhos Futuros:} Nesta seção são listados os trabalhos futuros para melhorar e/ou expandir a utilização da solução.
\end{itemize}