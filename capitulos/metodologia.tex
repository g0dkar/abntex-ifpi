% ----------------------------------------------------------
% Metodologia
% ----------------------------------------------------------
\chapter{Metodologia}

No desenvolvimento da aplicação utilizou-se de uma metodologia de desenvolvimento de software composta das seguintes fases: Levantamento de Requisitos, Modelagem, Implementação e Testes \cite{sommerville2003engenharia}.

\textbf{\textit{--- Assumo que a definição de levantamento de requisitos vem aqui? Talvez das 4 fases? E meio que entre nós: não exitem testes :P ---}}

O levantamento de requisitos foi feito através da análise do mercado de \emph{crowdfunding} nacional, do crescimento e popularização da prática dessa modalidade de investimento e tendencias internacionais em relação ao mercado de empreendimentos sociais. Os casos de uso resultantes do levantamento de requisitos foram:

\begin{lista}
  \item \textbf{Caso de Uso 01 - Doação}: O usuário se identifica com o trabalho de uma determinada instituição e quer fazer uma doação a mesma através da Internet;
  \item \textbf{Caso de Uso 02 - Comunicação com o Doador}: As instituições precisam de um canal para se comunicarem aos seus doadores e potenciais doadores;
  \item \textbf{Caso de Uso 03 - Acompanhamento das Doações}: É importante para a Instituição ter uma maneira de acompanhar a arrecadação de sua campanha e ter informações sobre quanto foi arrecadado e quando o dinheiro estará disponível;
  %\item \textbf{Caso de Uso 04 - Descobrimento de Instituições}: É importante para a Instituição ter uma maneira de acompanhar a arrecadação de sua campanha e ter informações sobre quanto foi arrecadado e quando o dinheiro estará disponível;
\end{lista}

As especificações dos casos de uso estão disponíveis no Anexo \ref{anexo:b} - Especificação de Casos de Uso.

Após o levantamento de requisitos e definição dos casos de Uso, se escolheu a metodologia \emph{Kanban} para guiar o processo de desenvolvimento de software. O \emph{Kanban} é uma metodologia para gestão de desenvolvimento que busca balancear demandas com base na capacidade de e para trabalhar. Membros do projeto "puxam" trabalho a medida que a capacidade de trabalho permite ao invés de ter trabalho vinculados a eles durante o processo \cite{wiki:Kanban}.

\textbf{\textit{--- O que mais botar aqui? ---}}





\section*{Resumo}
Neste capítulo foram apresentados os Casos de Uso levantados e o processo para implementação dos mesmos através de uma metodologia de desenvolvimento de software.

O próximo capítulo é dedicado a apresentação da solução implementada, detalhando funcionalidades, utilização e demais detalhes da implementação.