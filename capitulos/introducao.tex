% ----------------------------------------------------------
% Introdução
% ----------------------------------------------------------
\chapter{Introdução}

\section{Contextualização}
Em sua essência, Financiamento Coletivo é parte de um conceito mais amplo chamado Contribuição Colaborativa (ou Colaboração Coletiva - do inglês \emph{Crowdsourcing}). Em plataformas de Contribuição Coletiva utiliza-se do "coletivo" para se obter ideias, \emph{feedback} e soluções para problemas através de uma chamada ampla via Internet e a custo zero, ou bastante reduzidos.

Sites como \emph{The Mechanical Turk}\footnote{https://www.mturk.com} oferecem uma plataforma onde pessoas ou organizações podem colocar pedidos de micro-trabalhos, como votar na qualidade de traduções ou classificar vídeos em relação a seu conteúdo, e como recompensa a pessoa que fizer essas tarefas recebe micro-pagamentos de, por exemplo, U\$ 0,15 (quinze centavos de dólar) por cada vídeo classificado. Outros como o \emph{Kickstarter}\footnote{https://www.kickstarter.com} oferecem uma plataforma e rede social para arrecadar fundos para projetos normalmente relacionados a artes audiovisuais e atualmente é uma das mais prominentes plataformas de \emph{crowdfunding}. Em Novembro de 2017, nove das dez mais bem financiadas campanhas de \emph{crowdfunding} foram feitas via Kickstarter\footnote{Consultado em 31 jan. 2017 em http://crowdfundingblog.com/most-successful-crowdfunding-projects/}, juntas essas campanhas arrecadaram mais de U\$ 102 milhões.

Grandes empresas da Internet como Google, Amazon e Facebook estão apostando cada vez mais em soluções de \emph{Crowdsourcing} para inúmeras tarefas que necessitam de interação humana e para treinamento de Inteligências Artificiais. Um caso recente é o aplicativo Android Google \emph{Crowdsourcing} \cite{cnet-google-crowdsourcing} que dá aos usuários pequenas tarefas para auxiliar o aprendizado das inteligências artificiais por trás dos produtos Google como, por exemplo, reconhecimento de escrita, interpretação de textos em objetos reais captados via Google \emph{Street View}, avaliação de sugestões de traduções feitas através do serviço Google Tradutor, confirmação, correção ou complementação de informações sobre lugares reais cadastrados no Google Mapas (por exemplo se há ou não acesso a cadeirantes) e muitas outras micro-tarefas.

Desta nova forma de colaboração surgiu naturalmente um novo fenômeno: o Financiamento Colaborativo, ou \emph{Crowdfunding}. Ambos utilizam o poder de várias pessoas engajadas e pequenas contribuições de um grande número de pessoas para atingirem seus objetivos \cite{crowdfunding-culture}. Porém o fluxo de capital no \emph{Crowdfunding} é o contrário do \emph{Crowdsourcing}. Projetos que utilizam essa modalidade de financiamento pedem, através de plataformas online, pequenas contribuições financeiras de contribuidores individuais ou mesmo de investidores e fazem isso para que possam fazer algo mais pessoal como produção artística ou apoiar a produção de softwares como jogos. Assim é criado um novo modelo de investimento que circunver formas tradicionais de investimento como empréstimos junto a bancos, \emph{venture capital} (capital de risco) e afins \cite{belleflamme2010}.

A primeira plataforma de \emph{crowdfunding} a ter sucesso e ser responsável por iniciar de fato o mercado de \emph{crowdfunding} nos Estados Unidos foi o \emph{ArtistShare}\footnote{http://www.artistshare.com/} em 2003 \cite{freedman2015brief}. De autoria de Brian Camelio, um músico e programador de Boston, o \emph{ArtistShare} foi um \emph{website} onde músicos podiam buscar doações de seus fãs para possibilitar aos artistas a gravação e produção digital de música. Eventualmente o site se tornou em uma plataforma de financiamento coletivo para artistas audiovisuais, fotógrafos e músicos.

Seguindo essa tendencia, em 2005 Matt e Jessica Flannery idealizaram e lançaram o que é considerada a primeira plataforma para \emph{crowdfunding} social: Kiva\footnote{http://www.kiva.org}, uma agência de financiamento que utiliza \emph{crowdfunding} para prover micro crédito a empreendedores pobres em países em desenvolvimento mais notavelmente no Leste da África, India e Ásia Central. O Kiva é uma empresa sem fins lucrativos (\emph{non-profit}) e plataforma tecnológica que liga pessoas que têm mesmo que um pouco de dinheiro para ajudar e a vontade de ajudar a pessoas que necessitam de ajuda para ter um mínimo de qualidade de vida ou oportunidade \cite{flannery2007kiva}. A ideia começou quando os criadores do Kiva, durante uma viagem a África, conheceram o dono de uma peixaria na Etiópia que não tinha como melhorar seus lucros pois não tinha dinheiro para comprar uma passagem de ônibus e dependia de um atravessador para comprar peixes.

No Brasil o mercado de \emph{Crowdfunding} está ainda em seu começo, mas cresce a cada ano. Como exposto em  \cite{globo-financiamento} depois de passar mais de cinco anos procurando sem sucesso por investidores dispostos a financiar sua ideia, o arquiteto Márcio Cerqueira resolveu apostar em financiamento coletivo. A decisão foi fundamental para tirar do papel o Mola, espécie de \emph{Lego} que ajuda estudantes de arquitetura a entender melhor as estruturas de edifícios. O objetivo inicial era de levantar R\$ 50 mil, mas o projeto teve mais de 1.500 apoiadores e acabou arrecadando R\$ 600 mil, mais de dez vezes mais, algo que não teria obtido com grandes investidores. No Catarse\footnote{https://www.catarse.me}, uma das maiores plataformas nacionais de \emph{crowdfunding}, o volume arrecadado em 2016 foi de R\$ 16.2 milhões, um crescimento de 41\% em relação a 2015 \cite{catarse-retrospectiva2016}, e 134.827 pessoas apoiaram projetos na plataforma e desses 77.98\% apoiaram pela primeira vez (105.150 pessoas).

--- Fazer gancho para a justificativa ---



\section{Justificativa}
A situação atual das ONGs\footnote{Organizações Não-Governamentais}, como normalmente são chamadas organizações sem fins lucrativos, inclui dificuldades de várias ordens, mas as mais comuns e que muitas vezes impedem a iniciativa de continuar ou até mesmo começar são dificuldades em identificar fontes de financiamento e captar recursos. Elas enfrentam críticas sobre o papel que ocupam na economia e na sociedade, sua relação com o governo e as empresas \cite{GOUVEIA2007}. Além destes problemas muitas vezes ONGs têm problemas para captação de recursos junto a pessoas físicas, pois muitos acreditam que trabalho voluntário é o suficiente. No Brasil atualmente se vê um crescimento do trabalho voluntariado a um ponto que alguns autores \cite{fagundes2012repercussoes} consideram isso como influencia negativa a implementação de determinadas politicas sociais governamentais para diminuição da pobreza. Entende-se que se as próprias pessoas estão se mobilizando para resolver alguns problemas sociais, os mesmos se tornam menores e consequentemente menos recursos necessitem ser alocados para isso.

Apesar do crescimento das plataformas nacionais de \emph{crowdfunding}, da quantidade de projetos e de apoiadores essas plataformas têm taxas altas, comumente superiores a 10\%. Parte deste percentual são custos do \emph{gateway} de pagamentos utilizado pelo serviço como, por exemplo, o MoIP (\emph{Money Over IP}) que cobra de 3,49\% + R\$0,69 a 5,49\% + R\$0,69 por transação\footnote{Consultado em 27/01/2017 em https://moip.com.br/tarifas/}. Além desses custos, as plataformas nacionais não disponibilizam opção de escolha em relação a qual \emph{gateway} de pagamento o projeto deseja utilizar. O Catarse, por exemplo, utiliza o \emph{gateway} Pagar.me\footnote{Consultado em 27/01/2017 em http://crowdfunding.catarse.me/nossa-taxa}. Outro grande serviço nacional, o Kickante utiliza MoIP como \emph{gateway} de pagamento\footnote{Consultado em 27/01/2017 em https://www.kickante.com.br/termos/termos-de-uso, item 9.1.2}.

Ante os custos apresentados, as dificuldades envolvidas em outras formas de financiamento e as facilidades e potenciais benefícios este trabalho se propõe a criar uma plataforma de \emph{crowdfunding} de código aberto que acarrete o mínimo custo possível para os projetos financiados pela plataforma, focada para organizações de pequeno porte. Para este objetivo, um dos pontos principais e diferenciais da ferramenta é a possibilidade da escolha do projeto de qual \emph{gateway} de pagamento será usado para processar as doações ao projeto. Além disso, nenhum custo fora os custos embutidos pelo próprio \emph{gateway} de pagamento será impresso às doações dando assim uma maior margem ao projeto sob as doações recebidas.



\section{Objetivos}
\subsection{Objetivo Geral}
Prover um plataforma simples para facilitação de captação de recursos financeiros via Internet em modalidade \emph{Crowdfunding}.

\textbf{\textit{(nota ao professor: segundo a norma NBR6024 de 2012: ``Quando for necessário enumerar os diversos assuntos de uma seção que não possua título, esta deve ser subdividida em alíneas'' - Alíneas são esses ``pontos'' que usam letras e devem começar com minúsculas... Por alguma razão este é o único tipo de lista realmente suportado pelo abntex e sequer listado na documentação, acredito que deva ser o único especificado nas normas..)}}

\subsection{Objetivos Específicos}
\begin{lista}
  \item Realizar a análise, modelagem e implementação de uma solução em software web para facilitar a transferência de valores monetários entre pessoas e ONGs;
  \item Realizar a modelagem de uma arquitetura de aplicação web utilizando REST (\emph{Representational State Transfer});
  \item Selecionar um conjunto de tecnologias do ecossistema Java que melhor se adeque aos requisitos funcionais e não-funcionais do software.
\end{lista}



\section*{Resumo}
Neste capítulo foi apresentada uma contextualização sobre o problema tratado neste trabalho e a justificativa de tal assunto, que pode-se resumir como sendo necessidade de uma alternativa moderna e simplificada para captação de recursos para ONGs. Ao final, foram detalhados os objetivos gerais e específicos do trabalho.

Os próximos capítulos estão organizados da seguinte forma:

\begin{lista}
  \item \textbf{Fundamentação Teórica:} Neste capítulo são apresentados todos os conceitos teóricos utilizados no desenvolvimento da solução proposta no presente trabalho;
  \item \textbf{Metodologia:} Neste capítulo são detalhados os artefatos produzidos nas fases de levantamento de requisitos, modelagem e implementação do processo de software utilizado no desenvolvimento da solução;
  \item \textbf{Ajuda.Ai:} Capítulo dedicado a apresentação da solução implementada, detalhando funcionalidades, utilização e impressões de profissionais do empreendedorismo social quanto a ferramenta;
  \item \textbf{Conclusão:} Nesta seção é feita a conclusão do trabalho dado seus objetivos propostos;
  \item \textbf{Trabalhos Futuros:} Nesta seção são listados os trabalhos futuros para melhorar e/ou expandir a utilização da solução.
\end{lista}