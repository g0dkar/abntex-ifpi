%% Resumo
\begin{resumo}
Empreendimentos sociais no Brasil historicamente têm bastante dificuldade em buscar o financiamento necessário para exercer de forma plena as atividades que necessitam e frequentemente utilizam-se de campanhas de trabalho voluntário e doações de alimentos ou objetos usados.
  
Apesar da redução de custos que trabalho voluntário e doações trazem é inegável a necessidade de dinheiro para custos como eletricidade, água e medicamentos. Campanhas de arrecadação financeira comumente são discretas, como pedir o troco de uma compra como uma pequena doação para instiuições de caridade, ou complicadas para a instituição em si requerendo várias contas bancárias em diferentes bancos para arrecadar sem implicar em despesas bancárias ao doador. Além disso, algumas dessas opções não são viáveis para organizações menores ou que estão começando.

Este trabalho apresenta uma solução na forma de ferramenta online para crowdfunding social inspirada pelo serviço Kiva \cite{flannery2007kiva} e disponível para pequenas e médias organizações a fim que elas possam ter um canal virtual, seguro, direto e cômodo para chegar aos doadores e aos doadores para fazer com que doar a essas organizações seja tão simples quanto uma compra online.

  \vspace{\onelineskip}
  \noindent
  \textbf{Palavras-chaves}: financiamento coletivo. impacto social. empreendimento social.
\end{resumo}

%% Abstract (configurado para língua inglesa)
\begin{resumo}[Abstract]
	\begin{otherlanguage*}{english}
		Abstract do artigo. Em inglês obviamente. Deixa pra depois hehehe
        
		\vspace{\onelineskip}
		\noindent
		\textbf{Key-words}: crowdfunding. social impact. social entrepreneurship.
	\end{otherlanguage*}
\end{resumo}